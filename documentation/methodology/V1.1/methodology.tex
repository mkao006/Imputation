\documentclass[nojss]{jss}\usepackage{graphicx, color}
%% maxwidth is the original width if it is less than linewidth
%% otherwise use linewidth (to make sure the graphics do not exceed the margin)
\makeatletter
\def\maxwidth{ %
  \ifdim\Gin@nat@width>\linewidth
    \linewidth
  \else
    \Gin@nat@width
  \fi
}
\makeatother

\IfFileExists{upquote.sty}{\usepackage{upquote}}{}
\definecolor{fgcolor}{rgb}{0.2, 0.2, 0.2}
\newcommand{\hlnumber}[1]{\textcolor[rgb]{0,0,0}{#1}}%
\newcommand{\hlfunctioncall}[1]{\textcolor[rgb]{0.501960784313725,0,0.329411764705882}{\textbf{#1}}}%
\newcommand{\hlstring}[1]{\textcolor[rgb]{0.6,0.6,1}{#1}}%
\newcommand{\hlkeyword}[1]{\textcolor[rgb]{0,0,0}{\textbf{#1}}}%
\newcommand{\hlargument}[1]{\textcolor[rgb]{0.690196078431373,0.250980392156863,0.0196078431372549}{#1}}%
\newcommand{\hlcomment}[1]{\textcolor[rgb]{0.180392156862745,0.6,0.341176470588235}{#1}}%
\newcommand{\hlroxygencomment}[1]{\textcolor[rgb]{0.43921568627451,0.47843137254902,0.701960784313725}{#1}}%
\newcommand{\hlformalargs}[1]{\textcolor[rgb]{0.690196078431373,0.250980392156863,0.0196078431372549}{#1}}%
\newcommand{\hleqformalargs}[1]{\textcolor[rgb]{0.690196078431373,0.250980392156863,0.0196078431372549}{#1}}%
\newcommand{\hlassignement}[1]{\textcolor[rgb]{0,0,0}{\textbf{#1}}}%
\newcommand{\hlpackage}[1]{\textcolor[rgb]{0.588235294117647,0.709803921568627,0.145098039215686}{#1}}%
\newcommand{\hlslot}[1]{\textit{#1}}%
\newcommand{\hlsymbol}[1]{\textcolor[rgb]{0,0,0}{#1}}%
\newcommand{\hlprompt}[1]{\textcolor[rgb]{0.2,0.2,0.2}{#1}}%

\usepackage{framed}
\makeatletter
\newenvironment{kframe}{%
 \def\at@end@of@kframe{}%
 \ifinner\ifhmode%
  \def\at@end@of@kframe{\end{minipage}}%
  \begin{minipage}{\columnwidth}%
 \fi\fi%
 \def\FrameCommand##1{\hskip\@totalleftmargin \hskip-\fboxsep
 \colorbox{shadecolor}{##1}\hskip-\fboxsep
     % There is no \\@totalrightmargin, so:
     \hskip-\linewidth \hskip-\@totalleftmargin \hskip\columnwidth}%
 \MakeFramed {\advance\hsize-\width
   \@totalleftmargin\z@ \linewidth\hsize
   \@setminipage}}%
 {\par\unskip\endMakeFramed%
 \at@end@of@kframe}
\makeatother

\definecolor{shadecolor}{rgb}{.97, .97, .97}
\definecolor{messagecolor}{rgb}{0, 0, 0}
\definecolor{warningcolor}{rgb}{1, 0, 1}
\definecolor{errorcolor}{rgb}{1, 0, 0}
\newenvironment{knitrout}{}{} % an empty environment to be redefined in TeX

\usepackage{alltt}
\usepackage{url}
\usepackage[sc]{mathpazo}
\usepackage{geometry}
\geometry{verbose,tmargin=2.5cm,bmargin=2.5cm,lmargin=2.5cm,rmargin=2.5cm}
\setcounter{secnumdepth}{2}
\setcounter{tocdepth}{2}
\usepackage{breakurl}
\usepackage{hyperref}
\usepackage[ruled, vlined]{algorithm2e}
\usepackage{mathtools}
\usepackage{draftwatermark}











\title{\bf Statistical Working Paper on Imputation and Validation
  Methodology for the FAOSTAT Production Domain}

\author{Michael C. J. Kao\\ Food and Agriculture Organization \\ of
  the United Nations}

\Plainauthor{Michael. C. J. Kao} 
\Plaintitle{Working Paper on Imputation and Validation Methodology for
  the FAOSTAT Production Domain}
\Shorttitle{Imputation and Validation Methodology}

\Abstract{ 

  This paper proposes a new imputation method for the FAOSTAT
  production domain.  The proposal provides resolve to many of the
  shortcomings of the current approach, and offers a flexible
  framework to incorporate further information to improve performance.
  We first examine the factors that drive changes in production by
  commodity, after which a brief acount of the current approach and
  its shortcomings. Then, a description of the new methodology is
  provided, with a visual decomposition of the model and accompanying
  explanation.

}

\Keywords{Imputation, Linear Mixed Model, Agricultural Production, EM}
\Plainkeywords{Imputation, Linear Mixed Model, Agricultural Production, EM}

\Address{
  Michael. C. J. Kao\\
  Economics and Social Statistics Division (ESS)\\
  Economic and Social Development Department (ES)\\
  Food and Agriculture Organization of the United Nations (FAO)\\
  Viale delle Terme di Caracalla 00153 Rome, Italy\\
  E-mail: \email{michael.kao@fao.org}\\
  URL: \url{https://github.com/mkao006/Imputation}
}

%% \maketitle
%% \tableofcontents


\begin{document}

\section*{Disclaimer}
This Working Paper should not be reported as representing the views of
the FAO. The views expressed in this Working Paper are those of the
author and do not necessarily represent those of the FAO or FAO
policy. Working Papers describe research in progress by the author and
are published to elicit comments and to further discussion.

%% \tableofcontents
\section{Introduction}
Missing values are commonplace in the agricultural production domain,
stemming from non-response in surveys or a lack of capacity by the
reporting entity to provide measurement. Yet a consistent and
non-sparse production domain is of critical importance to Food Balance
Sheets (FBS), thus accurate and reliable imputation is essential and a
necessary requisite for continuing work. This paper addresses several
shortcomings of the current work and a new methodology is proposed in
order to resolve these issues and to increase the accuracy of
imputation.

The relationship between the variables in the production domain can be
expressed as:

\begin{equation}
  \label{eq:identity}
  \text{P}_t = \text{A}_t \times \text{Y}_t
\end{equation}


Where $P$, $A$ and $Y$ represent production, area harvested and yield,
respectively, indexed by time $t$. The yield is, however, unobserved and
can only be calculated when both production and area are
available. For certain commodities, harvested area may not exist or
sometimes it may be represented under a different context.


The primary objective of imputation is to incorporate all
available and reliable information in order to provide best estimates of
food supply in FBS.

\section{Background and Review of the Current Methodology}

There have been two classes of methodology proposed in the past in
order to account for missing values in the production domain. The
first type utilizes historical information and implements methods such
as linear interpolation and trend regression; while the second class
aims to capture the variation of relevant commodity
and/or spatial characteristics through the application of
aggregated growth rates. The imputation is carried out independently
on both area and production, with the yield calculated implicitly as
an identity.

Nevertheless, both approaches only utilize one dimension of
information and improvements can be obtained if information usage
can be married. Furthermore, these methods lack the ability to
incorporate external information such as vegetation indices,
precipitation or temperature that may provide valuable information and
enhance the accuracy of imputation.

Simulation results of the prior attempts indicate that linear
interpolation is a stable and accurate method but it lacks the
capability to utilize cross-sectional information. Furthermore, it
does not provide a solution for extrapolation where connection points
are not available. As a result, the aggregation method was then
implemented as it was found to provide a high coverage rate for
imputation with seemingly satisfactory performance.

In short, the aggregation imputation method computes the
commodity/regional aggregated growth of both area and production, the
growth rate is then applied to the last observed value of the
respective series. The formula of the aggregated growth can be
expressed as:

\begin{equation}
  \label{eq:aggregateGrowth}
  r_{s, t} = \sum_{c \in \mathbb{S}} X_{c, t}/\sum_{c \in \mathbb{S}} X_{c, t-1}
\end{equation}

Where $\mathbb{S}$ denotes the relevant set of products and countries
within the relevant commodity group and regional classification after
omitting the item to be imputed. For example, to compute the
\textit{country cereal aggregated growth} with the aim to impute wheat
production, we sum up all the production of commodities listed in the
cereal group in the same country excluding wheat. On the other hand,
to impute by \textit{regional item aggregated growth}, wheat
production data within the regional profile except the country of
interest are aggregated.\\


Imputation can then be computed as:
\begin{equation}
  \hat{X}_{c, t} = X_{c, t-1} \times r_{s, t}
\end{equation}
  

There are, however, several shortcomings of this methodology. The
Achilles heel lies in the fact that area and production are imputed
independently, cases of diverging area harvested and production have
been observed that result in inconsistency between trends as well as
exploding yields. The source of this undesirable characteristic is
nested in the computation of the aggregated growth rate. Owing to
missing values, the basket computed may not be comparable over time
and consequently results in spurious growth or
contraction. Furthermore, the basket to compute the changes in
production and area may be considerably different.

Finally, the methodology does not provide insight into the
underlying driving factors of production that are required to better understand the phenomenon and hence for
interpretation.



\section{Exploratory Data Analysis}




Before any modelling or statistical analysis, a grasp of the data is
essential. This section is devoted to some basic exploratory analysis
of the data in order to understand the nature of the series and their
drivers. First, let us explore the relationship between the identity
in equation \ref{eq:identity}. To make the relations hp clearer we have
log-transformed the data so the relationship becomes an additive one
rather than multiplicative.

\begin{equation}
  \label{eq:logIdentity}
  \log(P_t) = \log(A_t) + \log(Y_t)
\end{equation}



















\begin{knitrout}
\definecolor{shadecolor}{rgb}{0.969, 0.969, 0.969}\color{fgcolor}

{\centering \includegraphics[width=\linewidth,height=20cm]{figure/plot-decomposition} 

}



\end{knitrout}


In the above plots, the log of production, area and yield of a
specific commodity is depicted within each panel for comparison. Each
line represents a country and the production is the sum of area and
yield. The first noteworthy feature we observe from the relationship
between the series is that the level of production is mainly
determined by the level of harvested area, furthermore shocks are
typically reflected by a significant change in the area rather than
the yield. It is commonly believed that harvested area is stable and
predictable over time while vulnerable to shocks stemming from the
natural world.  Secondly, the range of the variability for yield is
very small in comparison to area, consistent with our intuition that
there are physical constraints on the potential yield of a crop within
a given size of area. The results are very similar even between
different commodities.

After exploring the relationship between the identity, let us delve
deeper into the constituents of production: area and yield. Depicted
below are area and yields for the same set of commodities but on an
original scale.
















\begin{knitrout}
\definecolor{shadecolor}{rgb}{0.969, 0.969, 0.969}\color{fgcolor}

{\centering \includegraphics[width=\linewidth,height=18cm]{figure/plot-area-yield} 

}



\end{knitrout}


We can first observe that area is in general much more stable and
smoother when compared to yield. The yield fluctuates from
year-to-year with correlation to a certain extent, which is more
prominently observed in wheat. This implies that there maybe
underlying factors such as climatic shocks, which may impact the yield
in different countries simultaneously. However, this characteristic is
not observed in the NES (none elsewhere specified) category which
suggests the impact of the factor is stronger within the same
commodity but weak in general.


The figures strongly suggests that both the trend and level of 
production is largely determined by  area harvested, but the
year-to-year fluctuation is driven by the yield, which may be
associated with changing climate conditions. The exploratory data analysis
reveals valuable insights into the nature of the time series, and
underpins the proposed model decomposition of variability and in
attributing the fluctuation to area and yield.





\section{Proposed Methodology}
In order to avoid identification problems and to capture the
correlation of yield between countries, we propose to impute the yield
and area in contrast to production and area. The additional advantage
of this approach, with well designed validation, almost guarantees
that the series will not diverge as is the case with the current
approach.


\subsection{Imputation for Yield}
The proposed model for imputing the yield is a linear mixed model, the
usage of this model enables all information available both
historical and cross-sectional to be incorporated. In addition,
proposed indicators such as the vegetation index, $\text{CO}_2$
concentration and other drivers can be tested and incorporated if
proven to improve predictive power.

The general form of the model can be expressed as:

\begin{align}
  \mathbf{y_i} &= \mathbf{X_i}\boldsymbol{\beta} +
  \mathbf{Z_i}\mathbf{b_i} + \epsilon_i \nonumber\\
  \mathbf{b_i} &\sim \mathbf{N_q}(\mathbf{0}, \boldsymbol{\Psi})\nonumber\\
  \epsilon_i &\sim \mathbf{N_{ni}}(\mathbf{0},
  \boldsymbol{\sigma^2}\boldsymbol{\Lambda_i})
\end{align}

Where the fixed component $\mathbf{X_i}\boldsymbol{\beta}$ models the
regional level and trend, while the random component of
$\mathbf{Z_i}\mathbf{b_i}$ captures the country specific variation
around the regional level.  More specifically, the proposed model for
FAOSTAT production has the following expression:

\begin{align}
  \label{eq:lmeImpute}
  \text{Y}_{i,t} &= \overbrace{\beta_{0j} + \beta_{1j}t}^{\text{Fixed
      effect}} + \overbrace{b_{0,i} + b_{1,i}t +
    b_{2,i, t}\bar{Y}_{j,t}}^{\text{Random effect}} + \epsilon_{i,t}
\end{align}

Where $Y$ denotes yield with $\bar{Y}$ being the grouped averaged
yield, $i$ for country, $j$ represents the designated regional
grouping and $t$ denotes time. Where the grouped averaged yield is
computed as:

\begin{equation}
  \label{eq:averageYield}
  \bar{Y}_{j, t} = \frac{1}{N_i}\sum_{i \in j} \hat{Y}_{i,t}
\end{equation}

However, as the grouped average yield is only partially observed
given the missing values, the average yield is estimated through the
EM-algorithm.


In essence, the imputation of the yield is based on the country
specific level and historical regional trend while accounting for
correlation between country and regional fluctuations. In contrast to
the previous methodology, where the full effect of the change is
applied, the proposed methodology measures the size of relationship
between the individual time series and the regional variability to
estimate the random effect for the country. Since both historical and
cross-sectional information are utilized, imputed values display
stable characteristics while reflecting changes in climatic
conditions.


To better understand the methodology, shown below is the sub-regional
decomposition with both the regional level (black line) and the
grouped yield (dark blue line) superimposed. The model assigns a
sub-regional level and trend depicted as the black line to each series
and models the correlation with the dark blue grouped yield series.


\begin{knitrout}
\definecolor{shadecolor}{rgb}{0.969, 0.969, 0.969}\color{fgcolor}

{\centering \includegraphics[width=\maxwidth,height=18cm]{figure/wheat-yield} 

}



\end{knitrout}




\subsection{Imputation for Harvested area}
After imputing the yield and computing area and production where
available, we then impute the area with linear interpolation and carry
forward the last observation when both production and area are not
available.

Following prior research and current investigation, we believe linear
interpolation is suitable because much of the harvested area data
exhibit extremely stable trends while linear interpolation yields a
satisfactory result. Despite the stability, shocks are sometimes
observed in the area series.  However, without a further understanding
of the nature and the source of the shocks, blindly applying the model
will introduce vulnerability rather than an anticipated improvement of
imputation performance. At the current stage, we have chosen to carry
forward and backward the latest available data where linear
interpolation is not applicable. The major advantage of this approach
is that if production ceases to exist and both production and area are
zero, we will not impute a positive value. Nevertheless, we are
continuing to explore the data and investigate superior methods which
may be applied to the imputation of area.


\begin{equation}
  \label{eq:linearInterpolation}
  \hat{A}_t = A_{t_a} + (t - a) \times \frac{A_{t_b} - A_{t_a}}{t_b - t_a}
\end{equation}

Then for values which we can not impute with linear interpolation, we
impute with the latest value.

\begin{equation}
  \label{eq:locf}
  \hat{A}_t = A_{t_{nn}}
\end{equation}


\section{Conclusion and Future improvements}
The aim of the paper has been to overhaul the current imputation
methodology, with a more consistent, yet better performing approach.

The proposed model demonstrates the ability to resolve issues such as
diverging area and production series and biased growth as a result of
missing values. Furthermore, the proposal takes into account the
attribute of incorporating relevant information as well as
establishing a flexible framework to accommodate additional
information.

This is, however, work in progress, as the technical teams are
continuing to collaborate in order to seek a deeper understanding of
the data in order to improve on the model. Application of the
state-space model might be a candidate to succeed in this regard, as
it allows for production, area and yield to be imputed simultaneously.

\section{Acknowledgement}
This work is supervised by Adam Prakash with assistance from Nicolas
Sakoff, Onno Hoffmeister and Hansdeep Khaira whom were crucial in the
development of the methodology. The author would also like to thank
the team members which participated in the first round of the
discussion providing valuable feedbacks. Finally, credits to Cecile
Fanton and Frank Cachia whom devoted their time to translate the paper
into French.



\section*{Annex 1: Geographic and classification}

The geographic classification follows the UNSD M49 classification at
\url{http://unstats.un.org/unsd/methods/m49/m49regin.htm}. The
definition is also available in the \code{FAOregionProfile} of the R
package \pkg{FAOSTAT}.

\section*{Annex 2: Pseudo Codes}

\begin{algorithm}
  \SetAlgoLined
  \BlankLine
  Initialization\;
  \Indp\Indp\Indp 
  $\hat{Y}_{i, t} \leftarrow f(Y_{i, t})$\;
  $\cal{L}_{\text{old}} = -\infty$\;
  $\cal{E}$ = 1e-6\;
  n.iter = 1000\;
  \Indm\Indm\Indm 
  
  \Begin{
      \For{i=1 \emph{\KwTo} n.iter}{
        E-step: Compute the expected group average yield\;
        \Indp\Indp\Indp 
        $\bar{Y}_{j, t} \leftarrow 1/N \sum_{i \in j} \hat{Y}_{i}$\;
        \Indm\Indm\Indm 
        
        M-step: Estimate the Linear Mix Model in \ref{eq:lmeImpute}\;
        \If{$\cal{L}_{\text{new}} - \cal{L}_{\text{old}} \ge
          \cal{E}$}{ $\hat{Y}_{i, t} \leftarrow \hat{\beta}_{0j} +
          \hat{\beta}_{1j}t + \hat{b}_{0i} + \hat{b}_{1i}t +
          \hat{b}_{2j}\bar{Y}_{j,t}$\; $\cal{L}_{\text{old}}
          \leftarrow \cal{L}_{\text{new}}$\; } \Else{ break } } }
    \caption{EM-Algorithm for Imputation}
    \label{alg:imputation}
\end{algorithm}
  
      

\begin{algorithm}[H]
  \SetAlgoLined
  \KwData{Production (element code = 51) and Harvested area (element
    code = 31) data}

  \KwResult{Imputation}
  
  \BlankLine
  Missing values are denoted $\emptyset$\;

  \BlankLine
  Initialization\;
  \Begin{
      \If{$A_t = 0 \land P_t \ne 0$}{
        $A_t \leftarrow \emptyset$\;
      }
      \If{$P_t = 0 \land A_t \ne 0$}{
        $P_t \leftarrow \emptyset$\;
      }
  }  
    
  \BlankLine  
  Start imputation\;
  \Begin{
      \ForAll{commodities}{
        
        (1) Compute the implied yield\;
        \Indp\Indp\Indp 
        $Y_{i,t} \leftarrow P_{i,t}/A_{i,t}$\;
        \Indm\Indm\Indm
                
        (2) Impute the missing yield with the imputation algorithm
        \ref{alg:imputation}\; \Indp\Indp\Indp
        
        %% $\hat{Y}_{i,t} \leftarrow \hat{\beta}_{0j} + \hat{\beta}_{1j}t
        %% + \hat{\beta}_{2j}\bar{Y}_{j,t} + \hat{b}_{0i} +
        %% \hat{b}_{1i}t$\;
        
        \Indm\Indm\Indm        
        
        \ForAll{imputed yield $\hat{Y}_{i, t}$}{
          \If{$A_t = \emptyset \land P_t \ne \emptyset$}{
            $\hat{A}_{i, t} \leftarrow P_{i, t}/\hat{Y}_{i, t}$\;
          }
          \If{$P_t = \emptyset \land A_t \ne \emptyset$}{
            $\hat{P}_{i, t} \leftarrow A_{i, t} \times \hat{Y}_{i, t}$\;
          }
        }
        
        (4) Impute area ($A_{i, t}$) with equation
        \ref{eq:linearInterpolation} then \ref{eq:locf}\;
        
        \ForAll{imputed area $\hat{A}_{i, t}$}{
          \If{$\hat{Y}_{i, t} \ne \emptyset$}{
            $\hat{P}_{i, t} \leftarrow \hat{A}_{i, t} \times \hat{Y}_{i, t}$\;
          }
        }
      }
  }
  \caption{Imputation Procedure}
\end{algorithm}

  
\section*{Annex 3: Supplementary Resources}

The data, code implementation and documentation can all be found and
downloaded from \url{https://github.com/mkao006/Imputation}. This
paper is generated on \today and is subject to changes and updates.
  
\begin{thebibliography}{9}
\bibitem{impWorkingPaper2011}
  Data Collection, Workflows and Methodology (DCWM) team,
  \emph{Imputation and Validation Methodologies for the FAOSTAT Production Domain}.
  Economics and Social Statistics Division,
  2011.
  
\bibitem{lairdWare1982}
  Nan M. Laird, James H. Ware,
  \emph{Random-Effects Models for Longitudinal Data}.
  Biometrics Volume 38, 963-974,
  1982.
  
\bibitem{rCore}
  R Core Team,
  \emph{A language and environment for statistical computing.}
  R Foundation for Statistical Computing, Vienna, Austria.
  ISBN 3-900051-07-0, URL http://www.R-project.org/,
  2013.
  
\bibitem{nlme}
  Jose Pinheiro, Douglas Bates, Saikat DebRoy, Deepayan Sarkar and the
  R Development Core Team,
  \emph{nlme: Linear and Nonlinear Mixed Effects Models.} 
  R package version 3.1-108.
  2013
  
\bibitem{rubin1976}
  Donald B. Rubin,
  \emph{Inference and Missing Data},
  Biometrika, Volume 63, Issue 3, 581-592,
  1976
\end{thebibliography}
  
  
\end{document}
