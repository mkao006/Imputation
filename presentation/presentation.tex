\documentclass{beamer}
\usepackage{amsfonts, amsmath, graphicx, verbatim, graphicx}
\usetheme{Warsaw}

\title{Imputation Methodology for FAOSTAT Production Domain}
\author{\it Michael. C. J.  Kao}
\institute{Food and Agriculture Organization \\of the United Nation}
\date{\today}


\AtBeginSection[]
{
  \begin{frame}<beamer>
    \frametitle{Outline for section \thesection}
    \tableofcontents[currentsection]
  \end{frame}
}

\begin{document}

\frame{
  \titlepage
}

\begin{frame}
  \frametitle{Outline}
  \tableofcontents
\end{frame}

\section{Introduction}
\frame{

  The aim of this presentation to give an over view of the current
  status of the newly proposed imputation methodology for the FAOSTAT
  production domain.\\

  \vspace{0.5cm}
  The work presented is the current status at the date of the
  presentation, and is subject to further changes.
}



\begin{frame}
  \frametitle{Why do we need imputation?}

  The agricultural production domain is integral to the compilation of
  Food Balance Sheets. In particular to estimate consistent food
  supplies, imputation is required to ensure that data are non-sparse.
  Owing to the potential impact of imputation when often data are
  missing, accuracy and reliability of food estimates cannot be
  compromised.

  \vspace{0.5cm}

  However, it must be recognized that imputation should only be used
  as a last resort.

\end{frame}

\begin{frame}  
  \vspace{0.5cm}
  The relationship of production and its components can be expressed as:

  \begin{equation}
    P_t = A_t \times Y_t
  \end{equation}
  
  Where $P_t$, $A_t$ and $Y_t$ denotes production, area
  harvested/animal slaughtered and yield/carcass weight, respectively,
  at time $t$.

\end{frame}  


\section{Current Methodology}

\begin{frame}

  The presently applied methodology aims to capture the variation of relevant
  commodity and/or geographic characteristics through the application
  of aggregated growth rates. a five-hierachy was designated represented 
  by:

  \begin{enumerate}
    \item Same country/commodity aggregate
    \item Sub-region aggregate/same commodity
    \item Sub-region aggregate/commodity aggregate
    \item Regional aggregate/same commodity
    \item Regional aggregate/commodity aggregate
  \end{enumerate}

\end{frame}

\begin{frame}

  In short, the aggregation imputation method computes the
  commodity/regional aggregated growth of both area and production,
  the growth rate is then applied to the last observed value. The
  formulae of the aggregated growth can be expressed as:
  
  \begin{equation}
    \label{eq:aggregateGrowth}
    r_{s, t} = \sum_{c \in \mathbb{S}} X_{c, t}/\sum_{c \in \mathbb{S}} X_{c, t-1}
  \end{equation}
  
  The imputation can then be computed as:
  \begin{equation}
    \hat{X}_{c, t} = X_{c, t-1} \times r_{s, t}
  \end{equation}
  
\end{frame}

\begin{frame}
  There are several shortcomings of the current methodology,
  \begin{itemize}
  \item Divergence of area and production, there are mainly two reasons for this.
    \begin{enumerate}
      \item Due to missing values, the aggregated growth can be heavily biased.
      \item The basket used to compute the aggregated growth rate is
        not the same over time and between area and production.
    \end{enumerate}
  \item Assumes perfect correlation between group and country series.
  \item Cannot support and incorporate additional information.
  \end{itemize}
\end{frame}

\section{Exploratory Data Analysis}

\frame{

  To give a brief demonstration, we have chosen Wheat to illustrate
  the properties and relationships between the time series.

  \vspace{0.5cm}

  The following graph illustrates the relationship between the
  production, area and yield.

}

\frame{
  \includegraphics[scale = 0.45]{wheatAreaYield}
}

\frame{  

  Now let us log transform the data so it becomes an additive
  relationshp.
  \begin{equation}
    \log(P_t) = \log(A_t) + \log(Y_t)
  \end{equation}

}


\frame{
  \frametitle{Area dictates the level and changes in the production}
  \includegraphics[scale = 0.45]{wheatIdentityBreakDown}
}


\frame{
  \frametitle{What is the data telling us?}

  What the data have shown is that the level, trend of the production
  is mainly determined by a smooth area occasionally affected by
  shocks, while the yield generates the variation from year-to-year
  reflecting climate or economic conditions.

  \vspace{0.5cm}
  
  This leads to the proposed methodology to estimate the year-to-year
  variation of yield while a stable method for area.

}


\section{Proposed Methodology}

\frame{

  First of all, we propose to impute the yield and area. This along
  with the restriction of the new model and the decomposition strategy
  almost guarantees that area and production will not diverge.\\
  \vspace{0.5cm}
  Second, instead of applying the changes directly, the model
  estimates the relationship between the country and the aggregated
  series and applies the factors accordingly.\\
  \vspace{0.5cm}  

  Finally the proposed model allows incorporation of additional
  information such as prices, precipitation and other information that
  may improve the accuracy of the imputation.

}


\subsection{Imputation for Area Harvested}
\frame{

  Currently we have adopted \textbf{linear interpolation} and
  \textbf{last observation carry forward} to impute area harvestd or
  what we called the naive imputation.
  
  \vspace{0.5cm}

  First the area harvested and in particular carcass weight per animal
  and trees displays extremely smooth behaviour and little
  year-to-year fluctuation and thus linear interpolation is
  suitable. Furthermore, a previous simulation study has shown linear
  interpolation gives best result in some cases.

  \vspace{0.5cm} 

  In addtion, "last observation carry forward" is useful when the last
  observed value is a true zero, we will not impute a positive value.

}

\subsection{Imputation for Yield}
\frame{
  \frametitle{Linear Mixed Model}

  To capture the co-movement of yield and model sub-regional
  differences, we have proposed to model the yield with a Linear Mixed
  Model (LME), which can be expressed as follows in matrix notation:

  \begin{align}
    \mathbf{y_i} &= \mathbf{X_i}\boldsymbol{\beta} +
    \mathbf{Z_i}\mathbf{b_i} + \epsilon_i \nonumber\\
    \mathbf{b_i} &\sim \mathbf{N_q}(\mathbf{0}, \boldsymbol{\Psi})\nonumber\\
    \epsilon_i &\sim \mathbf{N_{ni}}(\mathbf{0},
    \boldsymbol{\sigma^2}\boldsymbol{\Lambda_i})
  \end{align}
  
}

\frame{
  
  More specifically, the equation for the imputation is

  \begin{align}
    \label{eq:lmeImpute}
  \text{Y}_{i,t} &= \overbrace{\beta_{0j} + \beta_{1j}t +
    \beta_{2,i}\bar{Y}_{j,t}}^{\text{Fixed effect}} + \overbrace{b_{0,i} +
    b_{1,i}t}^{\text{Random effect}} + \epsilon_{i,t}
  \end{align}

  The average yield can be calculated as follow,

  \begin{align}
    \label{eq:averageYield}
    \bar{Y}_{j, t} = \sum_{i \in j} \omega_{i,t} Y_{i,t}
  \end{align}

  Which acts as a proxy to reflect the change in climatics conditions
  and other factors which can simultaneously affect multiple
  countries. 

}

\frame{

However, due to missing values, this quantity is not computed directly
from the raw data. The EM-algorithm is implemented for the estimation
for the unbiased average.

\vspace{0.5cm}

The average yield is only required if there are factors which can
simultaneously affect several countries within the region. That is,
the effect is only inlcluded when it improves the model.

\vspace{0.5cm}

The algorithm starts with the null model without the average effect,
then \textbf{AIC} is used as the decision criteria as whether the model with
average effect is needed.

}

\frame{
  \includegraphics[scale = 0.4]{wheatYieldSubregion}
}

\subsection{Imputation Procedure}
\frame{
  \includegraphics[page = 1, scale = 0.4]{wheatImputationStep}
}


\frame{
  \includegraphics[page = 2, scale = 0.4]{wheatImputationStep}
}


\frame{
  \includegraphics[page = 3, scale = 0.4]{wheatImputationStep}
}


\frame{
  \includegraphics[page = 4, scale = 0.4]{wheatImputationStep}
}


\frame{
  \includegraphics[page = 5, scale = 0.4]{wheatImputationStep}
}


\frame{
  \includegraphics[page = 6, scale = 0.4]{wheatImputationStep}
}



\section{Results}

\subsection{Individual imputation}

\frame{
  \includegraphics[page = 17, scale = 0.4]{checkWheatImputation}
}
\frame{
  \includegraphics[page = 18, scale = 0.4]{checkWheatImputation}
}


\frame{
  \includegraphics[page = 55, scale = 0.4]{checkWheatImputation}
}

\frame{
  \includegraphics[page = 61, scale = 0.4]{checkWheatImputation}
}

\frame{
  \includegraphics[page = 63, scale = 0.4]{checkWheatImputation}
}

\frame{
  \includegraphics[page = 98, scale = 0.4]{checkWheatImputation}
}


\subsection{Simulation Results}
\frame{
  \includegraphics[scale = 0.4]{wheatSimulationResult}
}

\section{Implementation}
\frame{ 

  Following the methodology, we will now present how the methodology
  will be integrated into the Food Balance Sheet cycle.

  Imputation does not create information, and it should be used as a
  last resort. Model can only be as good as data.

  \begin{enumerate}
    \item Receive data from production questionaire, countrySTAT or
      other official/semi-official source.
  
    \item Perform the imputation with data since 1980 to the most recent
    data, but only use the imputation for the years that are relevant
    to the preperation of the current Food Balance Sheet.
  
  \item Conduct expert assessment, and revise the imputation.
  \end{enumerate}
}


\section{Discussion}

\frame{
  Some on going work
  \begin{itemize}
    \item Investigate the performance of state-space model.
    \item Investigate ensemble models.
    \item Design a measure to quantify variance, in particular the
      frequency of fluctuation.
  \end{itemize}

}

\frame{

  The newly proposed methodology demonstrates the ability to resolve
  issues in the current methodology and extended to incorporate
  additional information.

  \vspace{0.5cm}
  We welcome any information which can enhance the performance of the
  imputation.
  \vspace{0.5cm}

  You can find all the data, codes implementation, documentation, and
  the wiki page at the following repository. You can also install the
  package following the instruction on the page.

  \vspace{0.5cm}
  \url{https://github.com/mkao006/Imputation}

}

\end{document}
