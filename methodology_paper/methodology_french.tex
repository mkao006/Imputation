\documentclass[nojss]{jss}\usepackage{graphicx, color}
%% maxwidth is the original width if it is less than linewidth
%% otherwise use linewidth (to make sure the graphics do not exceed the margin)
\makeatletter
\def\maxwidth{ %
  \ifdim\Gin@nat@width>\linewidth
    \linewidth
  \else
    \Gin@nat@width
  \fi
}
\makeatother

\IfFileExists{upquote.sty}{\usepackage{upquote}}{}
\definecolor{fgcolor}{rgb}{0.2, 0.2, 0.2}
\newcommand{\hlnumber}[1]{\textcolor[rgb]{0,0,0}{#1}}%
\newcommand{\hlfunctioncall}[1]{\textcolor[rgb]{0.501960784313725,0,0.329411764705882}{\textbf{#1}}}%
\newcommand{\hlstring}[1]{\textcolor[rgb]{0.6,0.6,1}{#1}}%
\newcommand{\hlkeyword}[1]{\textcolor[rgb]{0,0,0}{\textbf{#1}}}%
\newcommand{\hlargument}[1]{\textcolor[rgb]{0.690196078431373,0.250980392156863,0.0196078431372549}{#1}}%
\newcommand{\hlcomment}[1]{\textcolor[rgb]{0.180392156862745,0.6,0.341176470588235}{#1}}%
\newcommand{\hlroxygencomment}[1]{\textcolor[rgb]{0.43921568627451,0.47843137254902,0.701960784313725}{#1}}%
\newcommand{\hlformalargs}[1]{\textcolor[rgb]{0.690196078431373,0.250980392156863,0.0196078431372549}{#1}}%
\newcommand{\hleqformalargs}[1]{\textcolor[rgb]{0.690196078431373,0.250980392156863,0.0196078431372549}{#1}}%
\newcommand{\hlassignement}[1]{\textcolor[rgb]{0,0,0}{\textbf{#1}}}%
\newcommand{\hlpackage}[1]{\textcolor[rgb]{0.588235294117647,0.709803921568627,0.145098039215686}{#1}}%
\newcommand{\hlslot}[1]{\textit{#1}}%
\newcommand{\hlsymbol}[1]{\textcolor[rgb]{0,0,0}{#1}}%
\newcommand{\hlprompt}[1]{\textcolor[rgb]{0.2,0.2,0.2}{#1}}%

\usepackage{framed}
\makeatletter
\newenvironment{kframe}{%
 \def\at@end@of@kframe{}%
 \ifinner\ifhmode%
  \def\at@end@of@kframe{\end{minipage}}%
  \begin{minipage}{\columnwidth}%
 \fi\fi%
 \def\FrameCommand##1{\hskip\@totalleftmargin \hskip-\fboxsep
 \colorbox{shadecolor}{##1}\hskip-\fboxsep
     % There is no \\@totalrightmargin, so:
     \hskip-\linewidth \hskip-\@totalleftmargin \hskip\columnwidth}%
 \MakeFramed {\advance\hsize-\width
   \@totalleftmargin\z@ \linewidth\hsize
   \@setminipage}}%
 {\par\unskip\endMakeFramed%
 \at@end@of@kframe}
\makeatother

\definecolor{shadecolor}{rgb}{.97, .97, .97}
\definecolor{messagecolor}{rgb}{0, 0, 0}
\definecolor{warningcolor}{rgb}{1, 0, 1}
\definecolor{errorcolor}{rgb}{1, 0, 0}
\newenvironment{knitrout}{}{} % an empty environment to be redefined in TeX

\usepackage{alltt}
\usepackage{url}
\usepackage[sc]{mathpazo}
\usepackage{geometry}
\geometry{verbose,tmargin=2.5cm,bmargin=2.5cm,lmargin=2.5cm,rmargin=2.5cm}
\setcounter{secnumdepth}{2}
\setcounter{tocdepth}{2}
\usepackage{breakurl}
\usepackage{hyperref}
\usepackage[ruled, vlined]{algorithm2e}
\usepackage{mathtools}
\usepackage{draftwatermark}











\title{\bf Document de travail : Methodologie d'imputation et de
  validation pour le domaine de production de FAOSTAT}

\author{Michael C. J. Kao\\ Food and Agriculture Organization \\ of
  the United Nations}

\Plainauthor{Michael. C. J. Kao} 

\Plaintitle{Document de travail : m\'{e}thodologie d'imputation et de
  validation pour le domaine de production de FAOSTAT}

\Shorttitle{M\'{e}thode d'imputation et de validation}

\Abstract{ 
  
  Ce Papier pr\'{e}sente une nouvelle m\'{e}thode d'imputation
  destin\'{e}e au domaine de la production dans FAOSTAT.  Cette
  m\'{e}thode r\'{e}sout un nombre important de probl\`{e}mes
  soulev\'{e}s par l'approche actuelle, sa structure flexible permet
  d'incorporer de nouvelles informations et d'am\'{e}liorer ses
  performances.

  Nous examinons en premier lieu les facteurs d\'{e}terminant des
  changements de production par produits, puis donnons un bref aperçu
  de la m\'{e}thode actuelle et de ses limites. La nouvelle
  m\'{e}thodologie est ensuite d\'{e}crite, accompagn\'{e}e d'une
  d\'{e}composition du mod\`{e}le et de son explication.

}

\Keywords{Imputation, Mod\`{e}le lin\'{e}aire mixte
  g\'{e}n\'{e}ralis\'{e}, Production Agricole, EM}

\Plainkeywords{Imputation, Linear Mixed Model, Agricultural
  Production, EM}


\Address{
  Michael. C. J. Kao\\
  Economics and Social Statistics Division\\
  Economic and Social Development Department\\
  United Nations Food and Agriculture Organization\\
  Viale delle Terme di Caracalla 00153 Rome, Italy\\
  E-mail: \email{michael.kao@fao.org}\\
  URL: \url{https://github.com/mkao006/Imputation}
}

%% \maketitle
%% \tableofcontents


\begin{document}

%% \section*{Disclaimer}
%% This Working Paper should not be reported as representing the views of
%% the FAO. The views expressed in this Working Paper are those of the
%% author and do not necessarily represent those of the FAO or FAO
%% policy. Working Papers describe research in progress by the author and
%% are published to elicit comments and to further discussion.

%% \tableofcontents
\section{Introduction}
Les probl\`{e}mes de donn\'{e}es manquantes sont courants dans le
domaine de la production agricole. Ils peuvent être dus a une absence
de r\'{e}ponse de la part des entit\'{e}s pourvoyant les donn\'{e}es
ou une incapacit\'{e} de celles-ci \`{a} obtenir les informations.  Il
est cependant de premi\`{e}re importance, pour produire la balance
alimentaire de pouvoir compter sur un domaine de production
coh\'{e}rent et le plus complet possible. Une imputation pr\'{e}cise
et fiable en est donc un pr\'{e}-requis essentiel.

Ce papier cherche \`{a} cerner et d\'{e}passer un certain nombre de
limites de la m\'{e}thodologie actuelle et \`{a} am\'{e}liorer la
pr\'{e}cision de l'imputation en d\'{e}veloppant une nouvelle
m\'{e}thodologie.


La relation entre les variables du domaine de production peut être
exprim\'{e}e ainsi :
\begin{equation}
  \label{eq:identity}
  \text{P}_t = \text{A}_t \times \text{Y}_t
\end{equation}


O\`{u} $P$, $A$ et $Y$ repr\'{e}sentent respectivement la production, la
surface cultiv\'{e}e et le rendement, index\'{e}s par le temps $t$. Le
rendement est inobservable et peut seulement \^{e}tre calcul\'{e} quand la
production et la surface sont disponibles. Pour certains produits la
surface cultivable peut ne pas exister ou avoir une signification
diff\'{e}rente.

L'objectif de l'imputation est, en incorporant l'ensemble des
informations fiables utilisables, de fournir les meilleurs estimations
de la nourriture disponible pour permettre le calcul de la balance
alimentaire.


\section{Contexte et revue de la m\'{e}thodologie actuelle}
Deux cat\'{e}gories de m\'{e}thodologies ont \'{e}t\'{e} propos\'{e}es
par le pass\'{e} pour \'{e}valuer les donn\'{e}es manquantes dans le
domaine de production. Les m\'{e}thodologies appartenant \`{a} la
premi\`{e}re cat\'{e}gorie utilisent les s\'{e}ries de donn\'{e}es
historiques et appliquent des m\'{e}thodes d'interpolation et de
r\'{e}gression sur une tendance.  Celles appartenant \`{a} la seconde
cat\'{e}gorie basent l'imputation sur les taux de croissance des
produits d'int\'{e}rêt et/ou sur des agr\'{e}gations par
r\'{e}gion. L'imputation est men\'{e}e de mani\`{e}re ind\'{e}pendante
\`{a} la fois sur la surface cultiv\'{e}e et sur la production, tandis
que les rendements sont calcul\'{e}s de mani\`{e}re implicite.


Chacune de ces approches n'utilisent cependant qu'une dimension de
l'information. De nombreuses am\'{e}liorations peuvent \^{e}tre obtenues
en combinant les diff\'{e}rentes sources d'information et les
m\'{e}thodes cit\'{e}es plus haut.

De plus, ces m\'{e}thodes ne permettent pas d'incorporer d'autres
informations, comme les indices de v\'{e}g\'{e}tation, de
pr\'{e}cipitations, ou de temp\'{e}rature qui peuvent apporter une
information pr\'{e}cieuse et aider a am\'{e}liorer la pr\'{e}cision de
l'imputation.


Les r\'{e}sultats obtenus par les essais pr\'{e}c\'{e}dents indiquent
que l'interpolation lin\'{e}aire est une m\'{e}thode stable et
pr\'{e}cise. Elle ne permet cependant pas d'utiliser des
\'{e}chantillons transversaux, ni d'extrapoler lorsque les points de
connexion ne sont pas disponibles.

En cons\'{e}quent, la m\'{e}thode d'agr\'{e}gation a \'{e}t\'{e}
pr\'{e}f\'{e}r\'{e}e car elle permet d'atteindre un taux de couverture
\'{e}lev\'{e} pour l'imputation, et semble extrêmement performante.


Dans un premier temps, cette m\'{e}thode permet de calculer la
croissance agr\'{e}g\'{e}e de la production et de la surface par
produit et par r\'{e}gion.  Le taux de croissance est ensuite
appliqu\'{e} a la derni\`{e}re valeur observ\'{e}e dans la s\'{e}rie
concern\'{e}e. La formule de la croissance agr\'{e}g\'{e}e peut être
exprim\'{e} de la mani\`{e}re suivant :

\begin{equation}
  \label{eq:aggregateGrowth}
  r_{s, t} = \sum_{c \in \mathbb{S}} X_{c, t}/\sum_{c \in \mathbb{S}} X_{c, t-1}
\end{equation}

Où $\mathbb{S}$ r\'{e}f\`{e}re \`{a} l'ensemble des produits et pays
appartenant aux groupes de produits et de la classification
r\'{e}gionale concern\'{e}e, apr\`{e}s l'omission des donn\'{e}es
devant être imput\'{e}es.


Par exemple, pour calculer la \textit{croissance agr\'{e}g\'{e}e de
  production c\'{e}r\'{e}ali\`{e}re} pour un pays dans le but
  d'imputer la production de bl\'{e}, on additionne toute la
  production des produits appartenant au groupe de c\'{e}r\'{e}ales
  d'un même pays en excluant le bl\'{e}.

Pour imputer la production de bl\'{e} \`{a} l'aide d'un (\textit{
  indice r\'{e}gional de croissance agr\'{e}g\'{e}e }), les
donn\'{e}es de production du bl\'{e} sont agr\'{e}g\'{e}es \`{a}
l'int\'{e}rieur du profile r\'{e}gional, \`{a} l'exception du pays
concern\'{e}. \\


l'Imputation peut donc être calcul\'{e}e de la mani\`{e}re suivante
\begin{equation}
  \hat{X}_{c, t} = X_{c, t-1} \times r_{s, t}
\end{equation}
  
Il y a un certain nombre de limites \`{a} cette m\'{e}thodologie. Sa
faiblesse principale vient du fait que la production et la surface
sont estim\'{e}es de mani\`{e}re ind\'{e}pendante. Des cas de
divergence entre la production et la surface ont \'{e}t\'{e}
observ\'{e}s, r\'{e}sultant en incoh\'{e}rences entre les tendances,
ou en rendements bien trop \'{e}lev\'{e}es.


Ce probl\`{e}me prend sa source dans le calcul du taux de croissance
agr\'{e}g\'{e}e.

Du fait des donn\'{e}es manquantes, le panier calcul\'{e} peut ne pas
\^{e}tre comparable au cours du temps, induisant ainsi des erreurs dans le
calcul de la croissance ou de la contraction de la production. De
plus, les paniers permettant de calculer les changements de production
ou de surface cultiv\'{e}es peuvent \^{e}tre consid\'{e}rablement
diff\'{e}rents. Finalement, la m\'{e}thodologie ne donne aucun aper\c{c}u
des facteurs sous-jacents dirigeant la production, qui sont pourtant
n\'{e}cessaires \`{a} une meilleure compr\'{e}hension des
ph\'{e}nom\`{e}nes en jeux et donc \`{a} l'interpr\'{e}tation.


\section{Premi\`{e}re analyse de donn\'{e}es}




Avant qu'aucune mod\'{e}lisation ou analyse statistique ne soit faite,
un aperçu des donn\'{e}es est essentiel. Cette section est
d\'{e}di\'{e}e a l'exploration des donn\'{e}es afin de comprendre la
nature des s\'{e}ries et leurs d\'{e}terminants. En premier lieu, nous
explorerons la relation d\'{e}crite par l'\'{e}quation
\ref{eq:identity}. Pour simplifier, nous avons appliqu\'{e} aux
donn\'{e}es un logarithme, afin de transformer la relation
multiplicative en une relation additive.

\begin{equation}
  \label{eq:logIdentity}
  \log(P_t) = \log(A_t) + \log(Y_t)
\end{equation}



















\begin{knitrout}
\definecolor{shadecolor}{rgb}{0.969, 0.969, 0.969}\color{fgcolor}

{\centering \includegraphics[width=\linewidth,height=20cm]{figure/plot-decomposition} 

}



\end{knitrout}



Sur les graphiques ci-dessus, les log de la production, surface et des
r\'{e}coltes d'un produit sp\'{e}cifique sont trac\'{e} par panel pour
permettre la comparaison. Chaque ligne repr\'{e}sente un pays et la
production est la somme de la surface et du rendement. Le premier
aspect notable observ\'{e} ici est que le niveau de production est
principalement d\'{e}termin\'{e} par le niveau de surface
cultiv\'{e}e. Les chocs sur la production sont par ailleurs li\'{e}s a
des changements affectant la surface plus que les rendements. La
surface cultiv\'{e}e est habituellement consid\'{e}r\'{e}e comme
stable et pr\'{e}visible dans le temps, bien que vuln\'{e}rable aux
chocs naturels.


Le second aspect notable est que l'intervalle de variation du taux de
rendement est petit en comparaison de celui de la surface. Ceci est en
accord avec l'intuition qu'il existe des contraintes physique au
rendement potentiel d'une r\'{e}colte sur une surface donn\'{e}e. Ces
r\'{e}sultats ne varient pas selon les produits consid\'{e}r\'{e}s.


Nous allons maintenant explorer plus en d\'{e}tail l'\'{e}volution du
rendement et de la surface. Les graphiques ci-dessous repr\'{e}sentent
la surface et le rendement pour le même ensemble de produit, mais
cette fois sans transformation des donn\'{e}es.















\begin{knitrout}
\definecolor{shadecolor}{rgb}{0.969, 0.969, 0.969}\color{fgcolor}

{\centering \includegraphics[width=\linewidth,height=18cm]{figure/plot-area-yield} 

}



\end{knitrout}


Nous pouvons en premier lieu observer que les s\'{e}ries de la surface
cultiv\'{e}e sont en g\'{e}n\'{e}ral plus stables et lisses que celles
qui repr\'{e}sentent le rendement. Le rendement fluctue d'une
ann\'{e}e sur l'autre tout en pr\'{e}sentant une certaine
corr\'{e}lation, qui est plus durablement observ\'{e} dans la
s\'{e}rie du bl\'{e}. Ceci peut être expliqu\'{e} par des facteurs
sous-jacents, comme des facteurs climatiques, qui impacteraient les
rendements de diff\'{e}rents pays simultan\'{e}ment. Cependant cette
caract\'{e}ristique n'est pas observ\'{e}e dans la cat\'{e}gorie NES
(non sp\'{e}cifi\'{e}es ailleurs), ce qui suppose que l'impact de tels
facteurs est fort au sein d'un type de production mais faible entre
diff\'{e}rentes productions.


Les donn\'{e}es sugg\`{e}rent que la tendance et le niveau de la
production sont tr\`{e}s largement d\'{e}termin\'{e}s par la surface
cultiv\'{e}e, mais la variation d'une ann\'{e}e \`{a} l'autre est en
revanche d\'{e}termin\'{e}e par le rendement, qui peut \^{e}tre
associ\'{e} aux changements climatiques. L'analyse exploratoire des
donn\'{e}es nous \'{e}claire sur la nature de la s\'{e}rie
temporelle. Elle soutient le mod\`{e}le de d\'{e}composition de la
variance propos\'{e} en attribuant les fluctuations \`{a} la surface
et aux rendements.





\section{M\'{e}thodologie propos\'{e}e}

Afin d'\'{e}viter des probl\`{e}mes d'identification, et de capturer
la corr\'{e}lation des rendements entre diff\'{e}rents pays, nous
proposons d'imputer les rendements et la surface, et non la production
et la surface. Le second avantage de cette approche et qu'associ\'{e}e
a un system de validation, elle garantie que les s\'{e}ries ne
divergent pas comme elles le font dans l'approche actuelle.

\subsection{Imputation pour le rendement}
Le mod\`{e}le propos\'{e} pour estimer le rendement est un mod\`{e}le
lin\'{e}aire mixte, l'usage de ce mod\`{e}le permet d'incorporer \`{a}
la fois l'information transversale et l'information
historique. D'autres indicateurs, comme l'indexe de v\'{e}g\'{e}tation
$\text{CO}_2$, de concentration peuvent aussi être test\'{e}s et
incorpor\'{e}s si ils am\'{e}liorent la pr\'{e}vision.

La forme g\'{e}n\'{e}rale du mod\`{e}le peut être sp\'{e}cifi\'{e}e de
la mani\`{e}re suivante :
\begin{align}
  \mathbf{y_i} &= \mathbf{X_i}\boldsymbol{\beta} +
  \mathbf{Z_i}\mathbf{b_i} + \epsilon_i \nonumber\\
  \mathbf{b_i} &\sim \mathbf{N_q}(\mathbf{0}, \boldsymbol{\Psi})\nonumber\\
  \epsilon_i &\sim \mathbf{N_{ni}}(\mathbf{0},
  \boldsymbol{\sigma^2}\boldsymbol{\Lambda_i})
\end{align}

Où la composante fixe $\mathbf{X_i}\boldsymbol{\beta}$ d\'{e}signe le
niveau r\'{e}gional et la tendance , tandis que la composante
al\'{e}atoire $\mathbf{Z_i}\mathbf{b_i}$ capture la variation
sp\'{e}cifique du pays autour du niveau r\'{e}gional. Plus
sp\'{e}cifiquement, le mod\`{e}le propos\'{e} pour la production dans
FAOSTAT est le suivant :
\begin{align}
  \label{eq:lmeImpute}
  \text{Y}_{i,t} &= \overbrace{\beta_{0j} + \beta_{1j}t}^{\text{Fixed
      effect}} + \overbrace{b_{0,i} + b_{1,i}t +
    b_{2,i}\bar{Y}_{j,t}}^{\text{Random effect}} + \epsilon_{i,t}
\end{align}

O\`{u} $Y$ d\'{e}signe le rendement, $\bar{Y}$ d\'{e}signe le rendement
moyen du groupe, $i$ indique le pay , $j$ le groupe r\'{e}gional, et
$t$ le temps. La moyenne du groupe est calcul\'{e}e de la mani\`{e}re
suivante :
\begin{equation}
  \label{eq:averageYield}
  \bar{Y}_{j, t} = \frac{1}{N_i}\sum_{i \in j} \hat{Y}_{i,t}
\end{equation}

Cependant, comme le rendement moyen du groupe est seulement
partiellement observ\'{e} compte tenu des donn\'{e}es manquantes, le
rendement moyen est estim\'{e} grâce a l'algorithme EM (maximisation
de l'esp\'{e}rance).

L'estimation du rendement est bas\'{e}e sur le niveau sp\'{e}cifique
du pays et sur la tendance historique r\'{e}gionale, tout en tenant
compte de la corr\'{e}lation entre les pays et des variations
r\'{e}gionales.

Contrairement a la m\'{e}thodologie pr\'{e}c\'{e}demment utilis\'{e}e,
où la variation \'{e}tait appliqu\'{e}e enti\`{e}rement, la
m\'{e}thodologie propos\'{e}e mesure la taille de la relation entre la
s\'{e}rie individuelle et les variations r\'{e}gionales pour estimer
l'effet al\'{e}atoire du pays. Comme \`{a} la fois les donn\'{e}es
historiques et transversales sont utilis\'{e}s, les donn\'{e}es
estim\'{e}es pr\'{e}sentent des caract\'{e}ristiques stables tout en
refl\'{e}tant les changements climatiques.


Afin de mieux comprendre la m\'{e}thodologie, nous pr\'{e}sentons
ci-dessous le niveau r\'{e}gional (ligne noire), et le rendement moyen
des pays du groupe (ligne bleue fonc\'{e}e) sur le m\^{e}me graphique. Le
mod\`{e}le attribue a chaque s\'{e}ries une tendance et un niveau
r\'{e}gional (repr\'{e}sent\'{e} par la ligne noire), et mod\'{e}lise
la corr\'{e}lation avec la s\'{e}rie du rendement moyen r\'{e}gional
repr\'{e}sent\'{e} par ligne bleue fonc\'{e}e.


\begin{knitrout}
\definecolor{shadecolor}{rgb}{0.969, 0.969, 0.969}\color{fgcolor}

{\centering \includegraphics[width=\maxwidth,height=18cm]{figure/wheat-yield} 

}



\end{knitrout}




\subsection{Estimation pour la surface cultiv\'{e}e }

Apr\`{e}s avoir estim\'{e} le rendement, calcul\'{e} la surface
cultiv\'{e}e et la production quand cela \'{e}tait possible, nous
estimons la surface \`{a} l'aide d'une interpolation lin\'{e}aire et
r\'{e}pliquons la derni\`{e}re observation quand la production et la
surface ne sont pas disponibles.

D'apr\`{e}s de pr\'{e}c\'{e}dentes recherches et nos \'{e}tudes
actuelles, l'interpolation semble appropri\'{e}e car la surface
cultiv\'{e}e est caract\'{e}ris\'{e}e par des s\'{e}ries extr\^{e}mement
stables autour de leur tendance.


En d\'{e}pit de cette stabilit\'{e}, les chocs sont parfois
observ\'{e}s dans les s\'{e}ries de la surface cultiv\'{e}e.
Cependant, sans une compr\'{e}hension plus grande de la nature et de
la source de ces chocs, appliquer aveuglement le mod\`{e}le
n'am\'{e}liorerait pas la performance de l'estimation. Nous avons
choisi \`{a} ce stade de r\'{e}pliquer les derni\`{e}res donn\'{e}es
disponibles lorsque l'interpolation lin\'{e}aire n'est pas
applicable. L'avantage principal de cette approche est que si la
production cesse, les chiffres de la production et la surface
s'\'{e}tablissant a z\'{e}ro l'ann\'{e}e pr\'{e}c\'{e}dente, nous
n'imputerons pas une donn\'{e}e positive.

Nous continuons n\'{e}anmoins a explorer les donn\'{e}es et a
\'{e}tudier des m\'{e}thodes plus efficaces qui pourraient être
appliqu\'{e}es \`{a} l'estimation des donn\'{e}es manquantes pour la
surface.

\begin{equation}
  \label{eq:linearInterpolation}
  \hat{A}_t = A_{t_a} + (t - a) \times \frac{A_{t_b} - A_{t_a}}{t_b - t_a}
\end{equation}

Pour les donn\'{e}es manquantes pour lesquelles nous ne pouvons
imputer \`{a} l'aide de l'interpolation lin\'{e}aire, nous rempla\c{c}ons
par la derni\`{e}re valeur disponible.

\begin{equation}
  \label{eq:locf}
  \hat{A}_t = A_{t_{nn}}
\end{equation}


\section{Conclusion et am\'{e}liorations futures}

Le but de ce papier est de r\'{e}viser la m\'{e}thodologie actuelle,
et produire une m\'{e}thodologie plus pertinente et plus performante.


Le mod\`{e}le propos\'{e} permet de r\'{e}soudre des probl\`{e}mes
pos\'{e}s par les s\'{e}ries de donn\'{e}es de production et de
surface divergentes ou les biais dans le calcul croissance
r\'{e}sultant des donn\'{e}es manquantes. De plus, la proposition
offre la possibilit\'{e} d'incorporer l'information ad\'{e}quate tout
en maintenant un cadre souple permettant de traiter des informations
suppl\'{e}mentaires.

Les \'{e}quipes techniques continuent de collaborer afin
d'am\'{e}liorer le mod\`{e}le et de mieux comprendre les
donn\'{e}es. Un mod\`{e}le espace-\'{e}tat pourrait \^{e}tre un bon
candidat \`{a} cette m\'{e}thodologie car il permettrait \`{a} la
production, l'espace et le rendement d'être imput\'{e}s
simultan\'{e}ment.



\section*{Annex 1: G\'{e}ographie et classification}

La classification g\'{e}ographique suit la classification UNSD M49
\url{http://unstats.un.org/unsd/methods/m49/m49regin.htm}. La
d\'{e}finition est aussi disponible dans le \code{FAOregionProfile} du
package R \pkg{FAOSTAT}.


\section*{Annex 2: Code}


Les codes et les donn\'{e}es sont disponibles dans le fichier github 
\url{https://github.com/mkao006/Imputation}.

\begin{algorithm}
  \SetAlgoLined
  \BlankLine
  Initialization\;
  \Indp\Indp\Indp 
  $\hat{Y}_{i, t} \leftarrow f(Y_{i, t})$\;
  $\cal{L}_{\text{old}} = -\infty$\;
  $\cal{E}$ = 1e-6\;
  n.iter = 1000\;
  \Indm\Indm\Indm 
  
  \Begin{
      \For{i=1 \emph{\KwTo} n.iter}{
        E-step: Compute the expected group average yield\;
        \Indp\Indp\Indp 
        $\bar{Y}_{j, t} \leftarrow 1/N \sum_{i \in j} \hat{Y}_{i}$\;
        \Indm\Indm\Indm 
        
        M-step: Fit the Linear Mix Model in \ref{eq:lmeImpute}\;
        \If{$\cal{L}_{\text{new}} - \cal{L}_{\text{old}} \ge
          \cal{E}$}{ $\hat{Y}_{i, t} \leftarrow \hat{\beta}_{0j} +
          \hat{\beta}_{1j}t + \hat{b}_{0i} + \hat{b}_{1i}t +
          \hat{b}_{2j}\bar{Y}_{j,t}$\; $\cal{L}_{\text{old}}
          \leftarrow \cal{L}_{\text{new}}$\; } \Else{ break } } }
    \caption{EM-Algorithm for Imputation}
    \label{alg:imputation}
\end{algorithm}
  
      

\begin{algorithm}[H]
  \SetAlgoLined
  \KwData{Production (element code = 51) and Harvested area (element
    code = 31) data}

  \KwResult{Imputation}
  
  \BlankLine
  Missing values are denoted $\emptyset$\;

  \BlankLine
  Initialization\;
  \Begin{
      \If{$A_t = 0 \land P_t \ne 0$}{
        $A_t \leftarrow \emptyset$\;
      }
      \If{$P_t = 0 \land A_t \ne 0$}{
        $P_t \leftarrow \emptyset$\;
      }
  }  
    
  \BlankLine  
  Start imputation\;
  \Begin{
      \ForAll{commodities}{
        
        (1) Compute the implied yield\;
        \Indp\Indp\Indp 
        $Y_{i,t} \leftarrow P_{i,t}/A_{i,t}$\;
        \Indm\Indm\Indm
                
        (2) Impute the missing yield with the imputation algorithm
        \ref{alg:imputation}\; \Indp\Indp\Indp
        
        %% $\hat{Y}_{i,t} \leftarrow \hat{\beta}_{0j} + \hat{\beta}_{1j}t
        %% + \hat{\beta}_{2j}\bar{Y}_{j,t} + \hat{b}_{0i} +
        %% \hat{b}_{1i}t$\;
        
        \Indm\Indm\Indm        
        
        \ForAll{imputed yield $\hat{Y}_{i, t}$}{
          \If{$A_t = \emptyset \land P_t \ne \emptyset$}{
            $\hat{A}_{i, t} \leftarrow P_{i, t}/\hat{Y}_{i, t}$\;
          }
          \If{$P_t = \emptyset \land A_t \ne \emptyset$}{
            $\hat{P}_{i, t} \leftarrow A_{i, t} \times \hat{Y}_{i, t}$\;
          }
        }
        
        (4) Impute area ($A_{i, t}$) with equation
        \ref{eq:linearInterpolation} then \ref{eq:locf}\;
        
        \ForAll{imputed area $\hat{A}_{i, t}$}{
          \If{$\hat{Y}_{i, t} \ne \emptyset$}{
            $\hat{P}_{i, t} \leftarrow \hat{A}_{i, t} \times \hat{Y}_{i, t}$\;
          }
        }
      }
  }
  \caption{Imputation Process}
\end{algorithm}

%% \begin{thebibliography}{9}
%% \bibitem{impWorkingPaper2011}
%%   Data Collection, Workflows and Methodology (DCWM) team,
%%   \emph{Imputation and Validation Methodologies for the FAOSTAT
%%     Production Domain}.
%%   Economics and Social Statistics Division,
%%   2011  
%% \end{thebibliography}
  
  
\end{document}
