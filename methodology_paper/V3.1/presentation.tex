\documentclass{beamer}
\usepackage{amsfonts, amsmath, graphicx, verbatim, graphicx, hyperref,
  color}
\definecolor{UNBlue}{RGB}{91, 146, 229}
\setbeamercolor{structure}{fg=UNBlue}
\newcommand\Fontvi{\fontsize{6.5}{7.2}\selectfont}
\usetheme{Warsaw}

\title{Imputation and Adjustment of Production Domain}
\author{\it Michael C. J. Kao}
\institute{Food and Agriculture Organization \\ of the United Nations}
\date{}

\AtBeginSection[]
{
  \begin{frame}<beamer>
    \frametitle{Outline for section \thesection}
    \tableofcontents[currentsection]
  \end{frame}
}

\begin{document}

\frame{
  \titlepage
  \centering
  \includegraphics[scale = 0.2]{fao_logo.png}
}

\frame{
  \frametitle{Outline}
  \tableofcontents
}


\section{Extrapolation}
\frame{
  \frametitle{Introduction}
}
\subsection{Yield}
\frame{
  \frametitle{}
  % Start by looking at the country level
}

\frametitle{
  \frametitle{}
  % Plot the production by sub-region
}

\frame{
  \frametitle{}
  % illustrate a global fit where it can not adequately capture the
  % pattern.
}

\frame{
  \frametitle{}
  % illustrate a country-by-country fit and show how it can fail when
  % the number of points are small and the slope can be very volatile.
}

\frametitle{
  \frametitle{}
  % Illustrate how the linear mixed model solve the problem shown in
  % the previous two slide.
}

\frame{
  \frametitle{}
  % Talk a little about linear mixed model and why it works
}

\subsection{Production}

\frame{
  \frametitle{}
  % Illustrate the problem that country are very different and that
  % there are no regional information which we can take advantage of.
}

\frametitle{
  \frametitle{}
  % Plot the production by sub-region
}


%% Give several plots showing the different behaviour of the time
%% series.



\frame{
  \frametitle{What is the solution?}
  % Talk about how all time series behave differently, and also at the
  % same time that we may have very small number of observed points
  % for any complex modelling.
}

\frame{
  \frametitle{Ensemble Learning}
  % A simple description of what it is
}

\frame{
  \frametitle{Ensemble Learning}
  % Why it solves our problem.
}

\frame{
  \frametitle{Implementation}
  % A description of the ensemble method implemented
  %
  % (1) Base learner: Simple parametric functions.
  % (2) Combiner: non-trainable algebraic combiner.
}



\section{Balancing the Imputation}

\frame{
  \frametitle{}
  % Explain why we need this step to maintain the multivariate
  % relationship.

  % Also at the same time we make post correction based on the
  % relationship.
}

\frame{
  \frametitle{}
  % Illustrate that log(P) = log(A) + log(Y) and that we can use this
  % relationship to estimate local contribution of yield or area
  % harvested to production.
}


\frame{
  \frametitle{}
  % Decomposition of the production time series, illustrate a plot of
  % the spline basis.
}

\frame{
  \frametitle{}
  % Explain the significance of the weighting and how we plan to
  % estimate the weights from historical data.
}

\frame{
  \frametitle{}
  % Illustrate the result of the imputation and post-adjustments
}

\frame{
  \frametitle{Conclusion}
}


\frame{
  \frametitle{Further Improvements}
  % (1) Improve the weighting through mining historical accuracy of
  % different symbols.
  % (2) Change the decomposition from spline to wavelet decomposition.

}



\begin{frame}[allowframebreaks]
  \frametitle{Reference}
  \begin{thebibliography}{10}
  \end{thebibliography}
\end{frame}
  

\end{document}