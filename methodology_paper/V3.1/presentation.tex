\documentclass{beamer}
\usepackage{amsfonts, amsmath, graphicx, verbatim, graphicx, hyperref,
  color}
\definecolor{UNBlue}{RGB}{91, 146, 229}
\setbeamercolor{structure}{fg=UNBlue}
\newcommand\Fontvi{\fontsize{6.5}{7.2}\selectfont}
\usetheme{Warsaw}

\title{Imputation and Adjustment of the Production Domain}
\author{\it Michael C. J. Kao}
\institute{Food and Agriculture Organization \\ of the United Nations}
\date{}

\AtBeginSection[]
{
  \begin{frame}<beamer>
    \frametitle{Outline for section \thesection}
    \tableofcontents[currentsection]
  \end{frame}
}

\begin{document}

\frame{
  \titlepage
  \centering
  \includegraphics[scale = 0.2]{fao_logo.png}
}

\frame{
  \frametitle{Outline}
  \tableofcontents
}


\section{Introduction}
\frame{
  \frametitle{Introduction}

  The agricultural production domain is integral to the compilation of
  Food Balance Sheets. In particular to estimate consistent food
  supplies, imputation is required to ensure that data are non-sparse.

  Owing to the potential impact of imputation when often data are
  missing, accuracy and reliability of food estimates cannot be
  compromised.

}

\frame{
  \frametitle{Relationships}

  The relationship of production and its components can be expressed
  as:
  
  \begin{equation}
    \label{eq:pae}
    P_t := A_t \times Y_t \, \, \, \,\,\,\, P_t \ge 0, A_t \ge 0, Y_t > 0
  \end{equation}
  
  Where $P_t$ , $A_t$ and $Y_t$ denotes production, area harvested
  and yield, respectively, at time t.


  When production and area harvested are both missing, the yield is
  missing and needs to be imputed.
  
}

\frame{
  \frametitle{Outline}

  In this seminar, we will illustrate the proposing methodology for
  imputation of the production domain starting from yield, then
  production and area harvested. 

  Finally a balancing step to ensure the equation are satisfied and
  statistically valid.

}
  

\section{Imputation}
\subsection{Yield}
\frame{
  \frametitle{}
  % Start by looking at the country level
}

\frametitle{
  \frametitle{}
  % Plot the yield by sub-region
}

\frame{
  \frametitle{}
  % illustrate a global fit where it can not adequately capture the
  % pattern.
}

\frame{
  \frametitle{}
  % illustrate a country-by-country fit and show how it can fail when
  % the number of points are small and the slope can be very volatile.
}

\frametitle{
  \frametitle{}
  % Illustrate how the linear mixed model solve the problem shown in
  % the previous two slide.
}

\frame{
  \frametitle{}
  % Talk a little about linear mixed model and why it works
}

\subsection{Production and Area Harvested}

\frame{
  \frametitle{}
  % Illustrate the problem that country are very different and that
  % there are no regional information which we can take advantage of.
}

\frametitle{
  \frametitle{}
  % Plot the production by sub-region
}

\frame{
  \frametitle{}
  %% Give several plots showing the different behaviour of the time
  %% series. (Both production and area harvested)
}

\frame{
  \frametitle{Correlation}
  % Give several plots of production vs area harvested.
}
  

\frame{
  \frametitle{}
  We can see there is a strong linear correlation between production
  and area harvested.

  \vspace{0.1in}

  Which gives us a descent imputation when one of them exist.

  \vspace{0.1in}

  However, the core problem exist when both area harvested and
  production are missing.

}

\frame{
  \frametitle{What is the solution?}

  Oppose to yield, we have no cross-country information which we can
  utilize.

    % Talk about how all time series behave differently, and also at the
  % same time that we may have very small number of observed points
  % for any complex modelling.
}

\frame{
  \frametitle{What is Ensemble Learning?}

  Ensemble learning in its simplest sense, is to build multiple
  model/learner and combine them to obtain the final model or
  prediction.

  \vspace{0.1in}

  It consist of two components:
  \begin{enumerate}
    \item Build multiple model or learner
    \item Combine the model or predictions
  \end{enumerate}

}

\frame{
  \frametitle{Why Ensemble Learning?}

  Ensemble as described by Dietterich(2000) can mitigate the following
  three issues.

  \vspace{0.1in}
  
  \begin{itemize}
    \setlength{\itemindent}{1in}
    \item[Statistical:] Lack of data
    \item[Computational:] Model selection
    \item[Representational:] Complex model
  \end{itemize}

}

\frame{
  \frametitle{Implementation}
  The details of the ensemble implemented is describe here:

  \begin{itemize}
    \item {\textbf{Base learners:}}
      \begin{itemize}
        \item mean
        \item piecewise linear
        \item locally smooth linear
        \item logistic
        \item naive
      \end{itemize}
      \item {\textbf{Combiner:}}
        non-trainable algebraic combiner - Weighted sum rule
        \begin{equation}
          u_j(x) = \sum_{i=1}^N w_i d_{n, j}(x) \nonumber
        \end{equation}
  \end{itemize}
  Where the weights depends on the fit on the available data.

}



\section{Balancing the Imputation}

\frame{
  \frametitle{}

  After the initial imputation, we perform another step to balance the
  individual imputation.


  This is done not only to satisfy the production equation
  (\ref{eq:pae}), but to maintain the multivariate relationship between
  the series.

  Maintaining the multivariate relasionthip is an important property
  of imputation in particularly when they will be used for further
  analysis.

}

\frame{
  \frametitle{}

  During this step, we not only adjust imputations performed in the
  first step. But at the same time, we adjust values obtained from
  sources other than official/semi-official.

  This is done by assigning weights to data from difference
  sources. 

  Official and semi-official sources will have a full weight of 1,
  while other sources have currently been assigned a weight of 0.5.
  % Explain the significance of the weighting and how we plan to
  % estimate the weights from historical data.
}

\frame{
  \frametitle{}
  % Illustrate that log(P) = log(A) + log(Y) and that we can use this
  % relationship to estimate local contribution of yield or area
  % harvested to production.
}


\frame{
  \frametitle{}
  % Decomposition of the production time series, illustrate a plot of
  % the spline basis.
}


\frame{
  \frametitle{}
  % Illustrate the result of the imputation and post-adjustments
}

\frame{
  \frametitle{Conclusion}
}


\frame{
  \frametitle{Further Improvements}
  % (1) Improve the weighting through mining historical accuracy of
  % different symbols.
  % (2) Change the decomposition from spline to wavelet decomposition.

}



\begin{frame}[allowframebreaks]
  \frametitle{Reference}
  \begin{thebibliography}{10}
  \end{thebibliography}
\end{frame}
  

\end{document}